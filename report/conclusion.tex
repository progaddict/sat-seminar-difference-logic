\section{Conclusion}
\label{sec:conclusion}
DL logic is widely used to describe timing-related problems.
Although, it is possible to use simplex to solve SAT problem
of DL constraints, nevertheless there is a more efficient
graph-based algorithm.

The algorithm works on a constraint graph which represents
a conjunction of DL constraints.
The idea of the algorithm is based on the fact that
negative or zero length cycles in the constraint graph
correspond to a sequence of DL inequalities which,
if summed together, will produce a conflicting
inequality $ 0 < 0 $ or $ 0 \prec c $ where
$ c $ is a negative constant and $ \prec \; \in \{ <, \le \} $.
The inequalities, which correspond to the cycle,
are used to form the explanation for the SAT solver.

The ingenious thing about the algorithm is that it does not
enumerate all the cycles in a constraint graph but rather
detects any cycle in a dynamic admissible graph and
therefore achieves Bellman-Ford algorithm's complexity
$ O(|E| \cdot |V|) $.
Enumerating all cycles is prohibitively expensive in terms
of computational complexity because there can be exponentially
many of them in a graph.
