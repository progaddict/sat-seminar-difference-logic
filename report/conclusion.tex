\section{Conclusion}
\label{sec:conclusion}
Difference logic is widely used to describe timing-related problems.
Since difference logic is a special case of linear arithmetic logic,
it is possible to employ simplex to decide satisfiability
of difference logic constraints.
However, simplex has exponential complexity whereas the graph-based
algorithm, described in this report, has polynomial complexity.

The algorithm works on a constraint graph which represents
a conjunction of difference logic constraints.
The idea of the algorithm is
to find
a negative or zero length cycle
in the constraint graph.
If such a cycle exists
then it corresponds to a conflict,
the corresponding formula
is unsatisfiable
and the inequalities,
which correspond to the cycle,
form the explanation for the SAT solver.

The ingenious thing about the algorithm is that it does not
enumerate all the cycles in a constraint graph but rather
detects any cycle in a dynamic admissible graph and
therefore achieves Bellman-Ford algorithm's complexity
$ O(|E| \cdot |V|) $.
Enumerating all cycles is prohibitively expensive in terms
of computational complexity because there can be exponentially
many of them in a graph.
