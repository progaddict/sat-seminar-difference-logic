\section{Introduction}
\label{sec:introduction}
Introduction into the topic (1-2 pages).



\subsection{Difference Logic}
Definition of difference logic + examples of valid and invalid
formulas.
\begin{definition}[Difference Logic]
    \label{def:differenceLogic}
    Let $ \mathcal{P} = \{p_1, p_2, \ldots, p_n \} $ be ... (take definition from the main article~\cite{cotton2004some})
\end{definition}
A valid formula:
\begin{equation*}
    \phi_{\mathrm{valid}} = (x_1 - x_2 < 5)
    \land (x_1-x_3 \leq 7)
    \land (x_5 = -15)
    \land (x_1-x_5 < 10 \lor x_5-x_3 < -3)
\end{equation*}
and examples of invalid one is:
\begin{equation*}
    \phi_{\mathrm{invalid~1}} = (x_1 - x_2 < 5)
    \land (x_1-x_3 \leq \pi)
\end{equation*}
Because  $ \pi \notin \mathbb{Z} $.
Though, in practice (\ie for the real life applications) it is
usually possible to round $ \pi $ to
\eg 3.14 and then multiply both part of the equation by 100
and then introduce new variables:
\begin{equation*}
    \begin{aligned}
        x_1-x_3 \leq 3.14 \\
        100 x_1 - 100 x_3 \leq 314 \\
        x_1' - x_3' \leq 314 \\
        \mathrm{where} \; x_1' = 100 x_1 \;
        \mathrm{and} \; x_3' = 100 x_3
    \end{aligned}
\end{equation*}
Another one:
\begin{equation*}
    \phi_{\mathrm{invalid~2}} = x_1 + x_2 < 5
\end{equation*}
Because here we have addition.
Although, the reader can spot that it is possible to introduce
a new variable $ x_2' = -x_2 $ and turn
$ \phi_{\mathrm{invalid~2}} $ into a valid formula.
Yet another one:
\begin{equation*}
    \phi_{\mathrm{invalid~3}} = x_1 x_2 < 25
\end{equation*}
Because multiplication, division, exponentiation \etc are not
covered by the Difference Logic. Only difference of two variables
is valid.
Difference of more than two variable is not covered either:
\begin{equation*}
    \phi_{\mathrm{invalid~4}} = x_1 - x_2 - x_3 < 250
\end{equation*}
In such cases one should try to transform the "invalid" formula
into a valid "canonical" one by introducing new variables
(in the above's case it can be \eg $ x_2' = x_2 + x_3 $),
multiplying left and right side of the inequality by some
value \etc



\subsection{General Approach to SAT Solving}
Describe here the general approach to SAT solving presented
in the main paper~\cite{cotton2004some} (procedure Solve,
DPLL algorithm, generic scheme of tandem of SAT Solver
+ Theory Solver).
It is a good idea to include some picture/diagram describing
the DPLL approach (\eg a block scheme of the algorithm/procedure).
Maybe it is also a good idea to give a simple example.
Give definitions: implication graph, unique implication point.



\subsection{Approaches to Solve SAT of Difference Logic}
Shortly describe possible approaches (their core ideas)
to solve DL SAT problem.
According to the main article~\cite{cotton2004some}, they are:
\begin{itemize}
    \item The lazy approach
    \item The preprocessing approach
    \item Incremental approaches
\end{itemize}