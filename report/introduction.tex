\section{Introduction}
\label{sec:introduction}

DL is a special case of an LA logic in which
all LA constraints have the form $ x - y \prec c $ where
$ x $ and $ y $ are numerical variables, $ c $ is a constant
and $ \prec \in \{<, \leq \} $ is a comparison operator.
A more formal definition of DL is given
below~\cite{cotton2004some},~\cite[p.5]{mahfoudh2003verification}:
\begin{definition}[Difference Logic]
    Let $ \mathcal{B} = \{ b_1, b_2, \dots \} $ be a set of Boolean
    variables and $ \mathcal{X} = \{ x_1, x_2, \dots \} $ be a set
    of numerical variables. The difference logic over
    $ \mathcal{B} $ and $ \mathcal{X} $ is called
    $ DL(\mathcal{X},\mathcal{B}) $ and given by the following grammar:
    \begin{equation*}
        \phi \eqbydef b \; | \; (x - y < c) \; | \; \neg \phi \; | \;
        \phi \land \phi
    \end{equation*}
    where $ b \in \mathcal{B} $, $ x,y \in \mathcal{X} $ and
    $ c \in \mathbb{D} $ is a constant. The domain $ \mathbb{D} $ is
    either the integers $ \mathbb{Z} $ or the real numbers $ \mathbb{R} $.
    \\
    The remaining Boolean connectives
    $ \lor, \rightarrow, \leftrightarrow, \dots $ can be defined in
    the usual ways in terms of conjunction and negation.
\end{definition}
Examples of DL formulas are given below:
\begin{equation}
    \label{eq:example-1}
    \phi_1 = (p \lor q \lor r) \land (p \rightarrow (u - v < 3)) \land
    (q \rightarrow (v - w < -5)) \land (r \rightarrow (w - x < 0))
\end{equation}
\begin{equation}
    \label{eq:example-2}
    \phi_2 = (u-v < 1)
        \land (v-w < 5)
        \land (w-x \leq -3)
        \land (x-y < 1)
        \land (y-z \leq -5)
        \land (y-v \leq 0)
\end{equation}
\begin{equation}
    \label{eq:example-3}
    \phi_3 = (u-v < 1)
        \land (v-w < 5)
        \land (w-x \leq -3)
        \land (x-y < -3)
        \land (y-z \leq -5)
        \land (y-w < 4)
\end{equation}
