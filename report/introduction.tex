\section{Introduction}
\label{sec:introduction}



\subsection{Difference Logic}
\label{sec:difference-logic}
DL is a special case of an LA logic in which
all LA constraints have the form $ x - y \prec c $ where
$ x $ and $ y $ are numerical variables, $ c $ is a constant
and $ \prec \in \{<, \leq \} $ is a comparison operator.
A more formal definition of DL is given
below~\cite{cotton2004some, mahfoudh2003verification}:
\begin{definition}[Difference Logic]
    \label{def:differenceLogic}
    Let $ \mathcal{B} = \{ b_1, b_2, \dots \} $ be a set of Boolean
    variables and $ \mathcal{X} = \{ x_1, x_2, \dots \} $ be a set
    of numerical variables. The difference logic over
    $ \mathcal{B} $ and $ \mathcal{X} $ is called
    $ DL(\mathcal{X},\mathcal{B}) $ and given by the following grammar:
    \begin{equation*}
        \phi \eqbydef b \; | \; (x - y < c) \; | \; \neg \phi \; | \;
        \phi \land \phi
    \end{equation*}
    where $ b \in \mathcal{B} $, $ x,y \in \mathcal{X} $ and
    $ c \in \mathbb{D} $ is a constant. The domain $ \mathbb{D} $ is
    either the integers $ \mathbb{Z} $ or the real numbers $ \mathbb{R} $.
    \\
    The remaining Boolean connectives
    $ \lor, \rightarrow, \leftrightarrow, \dots $ can be defined in
    the usual ways in terms of conjunction and negation.
\end{definition}
Examples of DL formulas are given below:
\begin{equation}
    \label{eq:example-1}
    f_1 = (p \lor q \lor r) \land (p \rightarrow (u - v < 3)) \land
    (q \rightarrow (v - w < -5)) \land (r \rightarrow (w - x < 0))
\end{equation}
\begin{equation}
    \label{eq:example-2}
    f_2 = () \ land ()
\end{equation}
\begin{equation}
    \label{eq:example-3}
    f_3 = ()
\end{equation}



\subsection{Solving SAT Problem of Propositional Logic}
The satisfiability (SAT) checking problem 
Describe here the general approach to SAT solving presented
in the main paper~\cite{cotton2004some} (procedure Solve,
DPLL algorithm, generic scheme of tandem of SAT Solver
+ Theory Solver).
It is a good idea to include some picture/diagram describing
the DPLL approach (\eg a block scheme of the algorithm/procedure).
Maybe it is also a good idea to give a simple example.
Give definitions: implication graph, unique implication point.
\begin{definition}[Implication Graph]
    def
\end{definition}
\begin{definition}[Implication Point]
    def
\end{definition}
\begin{definition}[Unique Implication Point]
    def
\end{definition}



\subsection{Solving SAT Problem of Difference Logic}
Shortly describe possible approaches (their core ideas)
to solve DL SAT problem.
According to the main article~\cite{cotton2004some}, they are:
\begin{itemize}
    \item The lazy approach
    \item The preprocessing approach
    \item Incremental approaches
\end{itemize}