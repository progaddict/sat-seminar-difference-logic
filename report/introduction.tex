\section{Introduction}
Difference logic is a special case of linear arithmetic logic
and it is
defined~in~\cite{cotton2004some}~and~\cite[p.5]{mahfoudh2003verification}
as follows:
\begin{definition}[Difference Logic]
    \label{def:difference-logic}
    Let $ \mathcal{B} = \{ b_1, b_2, \dots \} $ be a set of Boolean
    variables and $ \mathcal{X} = \{ x_1, x_2, \dots \} $ be a set
    of numerical variables over a domain $ \mathbb{D} $.
    The domain $ \mathbb{D} $ is
    either the Integers $ \mathbb{Z} $
    or the Reals $ \mathbb{R} $.
    The difference logic over
    $ \mathcal{B} $ and $ \mathcal{X} $ is called
    $ DL(\mathcal{X},\mathcal{B}) $ and given by the following grammar:
    \begin{equation*}
        \phi \eqbydef b \; | \; (x - y \prec c) \; | \; \neg \phi \; | \;
        \phi \land \phi
    \end{equation*}
    where $ b \in \mathcal{B} $,
    $ x,y \in \mathcal{X} $,
    $ c \in \mathbb{D} $ is a constant
    and $ \prec \; \in \{ <, \leq \} $ is a comparison operator.

    The remaining Boolean connectives
    $ \lor, \rightarrow, \leftrightarrow, \dots $ can be defined in
    the usual ways in terms of conjunction $ \land $ and negation $ \neg $.
\end{definition}

The main difference between linear arithmetic logic
and difference logic is that in the latter
constraints contain only two variables.
A difference logic constraint is essentially
a comparison of a difference of those two variables
and a constant.
This form of constraints naturally emerges when
describing a delay between \eg starting times
of two processes or events,
which are described by the corresponding numerical variables.
It might be the reason why many timing related problems
can be described by difference logic.
Examples of difference logic formulas are given below:

\begin{equation}
    \label{eq:example-2}
    \phi_2 = (u-v < 1)
        \land (v-w < 5)
        \land (w-x \leq -3)
        \land (x-y < 1)
        \land (y-z \leq -5)
        \land (y-v \leq 0)
\end{equation}
\begin{equation}
    \label{eq:example-3}
    \phi_3 = (u-v < 1)
        \land (v-w < 5)
        \land (w-x \leq -3)
        \land (x-y < -3)
        \land (y-z \leq -5)
        \land (y-w < 4)
\end{equation}

Difference logic can also describe constraints
$ x \prec c $ and $ -x \prec c $.
One rewrites them as
$ x - 0 \prec c $ and $ 0 - x \prec c $ respectively
and then introduces a pseudo-variable~$ zero \notin \mathcal{X} $
instead of zero. An example:

\begin{equation}
    \begin{aligned}
        (v < 3) \land (-w \le -4) \\
        (v - 0 < 3) \land (0 - w \le -4) \\
        (v - zero < 3) \land (zero - w \le -4) \\
    \end{aligned}
\end{equation}

Algorithm~\ref{alg:goldberg-radzik} is compatible
with the pseudo-variable $ zero $ in the sense
that introducing $ zero $
does not break the reasoning, on which the algorithm
is based.
Since this algorithm constitutes the core of
the satisfiability checking of difference logic,
the above-described constraints can be
incorporated into a difference logic formula.

Constraints $ \pm x \succ c $ can be rewritten
as $ \mp x \prec -c $ and thus can also be described
by difference logic. An example:

\begin{equation}
    \begin{aligned}
        (v > -3) \land (-w \ge 4) \\
        (-v < 3) \land (w \le -4) \\
        (0-v < 3) \land (w-0 \le -4) \\
        (zero-v < 3) \land (w-zero \le -4)
    \end{aligned}
\end{equation}

Constraints $ x = c $ and $ x \ne c $
can be rewritten as
$ \neg ((x < c) \lor (x > c)) $ and
$ ((x < c) \lor (x > c)) $
respectively and thus can also be described
by difference logic.
An example:

\begin{equation}
    \begin{aligned}
        (v = -3) \land (w \ne 4) \\
        (\neg ((v < -3) \lor (v > -3))) \land ((w < 4) \lor (w > 4)) \\
        (\neg ((v < -3) \lor (-v < 3))) \land ((w < 4) \lor (-w < -4)) \\
        (\neg ((v-0 < -3) \lor (0-v < 3))) \land ((w-0 < 4) \lor (0-w < -4)) \\
        (\neg ((v-zero < -3) \lor (zero-v < 3))) \land ((w-zero < 4) \lor (zero-w < -4)) \\
    \end{aligned}
\end{equation}

One may employ even more sophisticated problem-specific
transformations like in the following example:

\begin{equation}
    \begin{aligned}
        (u - 2w + v > 3) \land (2w - v - u \ge 2) \\
        (u - w - w + v > 3) \land (w - v - u + w \ge 2) \\
        ((u - w) - (w - v) > 3) \land ((w - v) - (u - w) \ge 2) \\
        \mathrm{introduce \; new \; variables} \; x = w - v \; \mathrm{and} \; y = u - w \\
        (y - x > 3) \land (x - y \ge 2) \\
        (x - y < -3) \land (y - x \le -2) \\
    \end{aligned}
\end{equation}

A job scheduling problem is a motivation
behind difference logic.
The problem is formulated as follows.
Given $ N $ jobs with processing times
$ \tau_1, \dots, \tau_N $
and $ M $ identical machines,
each of which can process only one job at any given time moment,
the following question must be answered.
Is it possible to schedule the jobs on the machines
such that the overall processing time will
not exceed $ T $? An example setting:

\begin{equation}
    \begin{aligned}
        N = 4, M = 2, T = 6.5 \\
        \tau_1 = 1.6, \tau_2 = 1.1, \tau_3 = 4.6, \tau_4 = 5.1 \\
    \end{aligned}
\end{equation}

Let $ p_{mj} = True $ iff job $ j $ is scheduled on machine
$ m $.
Let also $ t_j $ be the start time of the job $ j $.
Then the job scheduling problem for the given example setting
can be encoded by the following formula:

\begin{equation}
    \label{eq:job-scheduling-example}
    \begin{aligned}
    \phi &= \bigwedge_{j=1}^{4} (p_{1j} \lor p_{2j}) \quad \land \\
    & \mathrm{each \; task \; is \; executed \; on \; at \; least \; one \; machine} \\
    & \bigwedge_{j=1}^{4} ((p_{1j} \rightarrow \neg p_{2j}) \land (p_{2j} \rightarrow \neg p_{1j})) \quad \land \\
    & \mathrm{each \; task \; can \; be \; scheduled \; on \; one \; machine\; only } \\
    & \bigwedge_{j=1}^{4} (t_j \ge 0) \; \land \; \bigwedge_{j=1}^{4} (t_j \le T-\tau_j) \quad \land \\
    & \mathrm{general \; time \; constraints} \\
    & \bigwedge_{m=1}^{2} \bigwedge_{i=1}^{3} \bigwedge_{j=i+1}^{4} ((p_{mi} \land p_{mj}) \rightarrow ((t_i-t_j \le -\tau_i) \lor (t_j-t_i \le -\tau_j))) \\
    & \mathrm{jobs} \; i \; \mathrm{and} \; j \; \mathrm{must \; not \; overlap \; if \; scheduled \; on \; the \; same \; machine} \; m \\
    \end{aligned}
\end{equation}
Where $ \tau_j $ and $ T $ are real constants
(the $ c \in \mathbb{D} $
in Definition~\ref{def:difference-logic},
$ \mathbb{D} = \mathbb{R} $),
$ p_{mj} $ are boolean variables
from the set $ \mathcal{B} $
in Definition~\ref{def:difference-logic}
and $ t_j $ are real variables
from the set $ \mathcal{X} $
in Definition~\ref{def:difference-logic}
($ 1 \le m \le 2, 1 \le j \le 4 $).

All the linear arithmetic constraints
in Equation~\ref{eq:job-scheduling-example}
can be expressed in difference logic.
Thus this problem can be solved by
a difference logic SAT solver.

The rest of the report is organized as follows.
Chapter~\ref{sec:preliminaries} gives theoretical background
on SAT checking.
Chapter~\ref{sec:topic} describes a graph-based algorithm to solving
SAT problem of DL.
Chapter~\ref{sec:conclusion} draws a conclusion.